\documentclass[11pt]{article}
\usepackage[english]{babel}
\usepackage{listings}
\usepackage{xcolor} 
\usepackage{courier}
\usepackage{hyperref}

\lstset{
language=sh
,breaklines=true
,frame=single
,tabsize=2
,basicstyle=\ttfamily
,showstringspaces=false
,numbers=left
,numberstyle=\tiny\color{gray}}

\setlength\parindent{0pt}

\newcommand{\io}[1]{\textbf{#1}}

\title{SORTES - Clock}
\author{Dieter Castel (0256149) \\ Jonas Devlieghere (0256709)}
\date{\today}
 
\begin{document}
\maketitle
\newpage

\tableofcontents
\newpage

\section{User Documentation}
Two buttons allow the user to interact with the device. 
The bottommost is \io{button 1} and is used to browse trough a items such as numbers or menu items.
The topmost is \io{button 2} and is generally used to confirm your selection or to enter configuration mode.

\subsection{Configuration Mode}
To enter configuration mode from regular mode (i.e. the clock is displayed) press \io{button 2}. 
A menu will display allowing you to change the current time and alarm or quit configuration mode.
When the devices powers on, you will automatically enter configuration mode since the current time has not been set. 
This means you will not have to press \io{button 2} to enter configuration mode. 

\subsection{Setting the Clock}
To configure the current time, press \io{button 2} to enable configuration mode if not yet enabled. 
Press \io{button 1} until \io{Set time?} is displayed. 
Confirm this choice by pressing \io{button 2}.
You will be able to configure the clock in 3 simple steps respectively setting the hours, minutes and seconds. 
Use \io{button 1} to increase the value of each property. 
The input will automatically wrap around when the maximum value is reached.
For example, pressing {button 1} when the current hour value is 23 will yield a value of 0. 
Confirm each input value by pressing {button 2}. When all values are set you will return to the configuration menu.

\subsection{Setting the Alarm}
Configuring the alarm is almost identical to configuring the current time. Press \io{button 2} to enter configuration mode, navigate to \io{Set alarm?} by pressing \io{button 1} and press \io{button 2} once more. 
Follow the steps mentioned above to configure the alarm time as desired.

\section{System Documentation}
We provided a makefile to compile the source code. Run the following command:
\begin{lstlisting}
$ make clock 
\end{lstlisting}
To deploy the clock.hex file to the PIC a shell script is available. The script will start tftp and wait for input from the user. 
\begin{lstlisting}
$ ./deploy.sh 
\end{lstlisting}
Enter the following command but do not press return just yet. Reset the PIC and wait for the corresponding LED on the router to light up, then press return.
\begin{lstlisting}
put clock.hex
\end{lstlisting}
When all of this succeeds, you'll see something like this. The amount of time and bytes may differ.
\begin{lstlisting}
$ make clock 
#################### BUILD INIT ####################
sdcc -mpic16 -p18f97j60 -L /usr/local/lib/pic16 -llibio18f97j60.lib -llibdev18f97j60.lib -llibc18f.lib -L include objects/clock.o objects/LCDBlocking.o objects/newtime.o
message: using default linker script "/usr/local/share/gputils/lkr/18f97j60.lkr"
#################### BUILD DONE ####################
$ ./deploy.sh 
starting tftp to 192.168.97.6
put clock.hex
tftp> tftp> Sent 35956 bytes in 2.1 seconds
\end{lstlisting}

\section{System Design}

\subsection{Specifications}
The program provides an configurable clock with alarm function. At startup the user can set the clock and optionally the alarm using two button on the device. Once the time is set an orange led blinks each half a second. When the current time equals the alarm time an alarm goes of. This means two red LEDs start blinking one second on, one second of during the next 30 seconds. During operation both the alarm and time can be adjusted. Setting an alarm does not influence the current time.

\subsection{Structural Choices}
\subsubsection{Timer}
To effectively measure time we are using a hardware timer provided by the PIC. This timer will interrupt when its buffer overflows. We prefer this method over working with software delays because of it's increase in accuracy. Software timers are more easily influenced by (possible unknown) software implementations (i.e. arithmetics).
\\\\
The timer can be operated in either 8 or 16 bit mode. This marks the length of its buffer and thus the delay between a software interrupt arises. Operating in 16 bit mode opposed to 8 bit may increase accuracy between interrupts but on the other hand might introduce a too rough granularity. A software counter is used to count the amount of interrupts between the elapse of one second. Empirical testing has shown us that this last effect is indeed a problem. Therefore, we have increased the amounts of interrupts and have chosen to operate the timer in 8 bit mode.

\subsubsection{Buttons}
Buttons are too implemented using interrupts. When a button is pressed a dummy register is set to true. The registers are only accessed using a ``read-and-clear'' operation. This ensures that every button press is read atomically. The alternative is reading the actually register. This approach requires a certain delay between two read operations to ensure a single press is not interpreted as more than one. Our approach does not have this problem. The disadvantage however consists of the impossibility to press a button continuously e.g. for increasing a value without releasing the button.

\subsubsection{Time Structure}
Storing the current time can be done in a number of ways. We could've kept a counter of nano- or milliseconds since the beginning of time, the start of the device or since midnight. We chose for a different approach in order to save space. A structure with 3 fields (hours, minutes and seconds) is used to store the current time. This had the additional advantage of introducing a certain level of abstraction. Furthermore, this structure also represents the time of the alarm.  

\subsection{Calibration}
In order to calibrate our clock we needed to find out how many interrupts occur over the course of one second.
At first instance, when we were still using the 16 bit timer, we found that the correct value was between 95 en 96 interrupts per second.
This meant our counting system was to course-grained.
This is when we switched to operating the timer in 8 bit mode, allowing more interrupts to occur.  
\\\\
Using a binary search to obtain the optimal value, we found that \textbf{24407} interrupts correspond to one second.
In table \ref{tab:bs} you can see how we acquired the optimal value. 
The first upper and lower values were found by using the counter in 16-bit mode.
Using that mode we found that one second falls between $95$ and $96$ overflows of the counter.
Since the counter in 8-bit mode overflows $256$ times faster we find the initial values: $24320$ ($256*95$) and$24576$ ($256*96$).

\begin{table}
	\centering
	\begin{tabular}{|r || c | c | c | c |}
		\hline
		Test value & Lower Value & Middle & Upper Value & Result \\ \hline
		/ & 24320 &  24448 & 24576 & / \\ \hline
		24448 & 24320 & 24384 & 24448 & fast \\ \hline
		24384 & 24384 & 24416 & 24448 & slow \\ \hline
		24416 & 24384 & 24400 & 24416 & fast \\ \hline
		24400 & 24400 & 24408 & 24416 & GOOD \\ \hline
	\end{tabular}
	\caption{The approximation of the amount of counter overflows in one second by binary search.}
	\label{tab:bs}
\end{table}

\subsection{Technical Peculiarities}
If you want to make use of button \textbf{1} you need to make use of register ``INTCON3bits.INT\textbf{3}F''.
For button \textbf{2} instead, you need to use the register ``INTCON3bits.INT\textbf{1}F''. This is also poorly documented.
\\\\
For using structures malloc() is needed (which is included in the C library for the PIC16) but what is not included is actually creating the stack. That's why you have to manually assign space for the stack. How this is done can be seen on line 7 of listing \ref{time.c} on page \pageref{time.c}.
\\\\
Another headache was setting the correct size of the stack. Sometimes the stack was too large, sometimes it was too small. In both cases it caused the time structure to represent garbage. Fortunately this was fairly easy visible when displaying the current time. Adjusting the size on the other hand was not, reasoning didn't seem to get us anywhere so we ended up using trail and error to determine a workable amount.


\appendix

\section{Source Code}
\lstinputlisting[language=C, caption={strings header file}, label=strings.h]{../src/strings.h}
\lstinputlisting[language=C, caption={clock body file}, label=clock.c]{../src/clock.c}
\lstinputlisting[language=C, caption={clockio header file}, label=clockio.h]{../src/clockio.h}
\lstinputlisting[language=C, caption={clockio body file}, label=clockio.c]{../src/clockio.c}
\lstinputlisting[language=C, caption={time header file}, label=time.h]{../src/time.h}
\lstinputlisting[language=C, caption={time body file}, label=time.c]{../src/time.c}
\end{document}
